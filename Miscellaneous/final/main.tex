\documentclass[12pt]{article}
\usepackage[utf8]{inputenc}
\usepackage{style/rsltn}
\usepackage{style/themes/tundra}

\title{An Exercise for Mason}
\author{Robin Truax}
\date{Summer 2023}

\begin{document}

\maketitle

\begin{proposition}[Exercise 1]\label{prop:Exercise_1}
    Suppose that $G$ is a compact, connected topological group, and $U$ is a nonempty open subset of $G$.
    Then $U^k = G$ for some finite $k$.
\end{proposition}
\begin{proof}
    Consider the smallest subgroup $H$ generated by $U$.
    Since $H = \bigcup_{n \in \Z}U^n$, and $U^n$ is open for each $n \in \Z$, $H$ is an open subgroup of $G$.
    Thus, $H$ is a closed subgroup of $G$ whence by connectedness $H = G$.
    It then follows that $\bigcup_{n \in \Z}U^n$ is an open subcover of $G$, so it has a finite subcover $U^{-n_1} \cup \cdots \cup U^{-n_k} \cup U^{m_1} \cup \cdots \cup U^{m_l}$.
    Now, both $U^{m_1} \cup \cdots \cup U^{m_l}$ and $U^{-n_1} \cup \cdots \cup U^{-n_k}$ are open, and they cover $U$.
    Thus, either they are disjoint, or they intersect.
    \\

    In the former case, $U^{m_1} \cup \cdots \cup U^{m_l}$ is open and closed, and thus is equal to $G$.
    Thus, in particular, $1 \in U^m$ for some $m$.
    In the latter case, there exists some $m_i,n_j$ such that $U^{m_i} \cap U^{-n_j}$ has an intersection.
    It then follows that $1 \in U^{m_i + n_j}$.
    Again, $1 \in U^m$ for some $m$.
    \\

    Now, let $W = U^m \cap U^{-m}$; $W$ is symmetric and open, so the subgroup $J$ generated by $W$ is equal to $\bigcup_{k \in \Z^+} W^k$.
    Since $J$ is the union of open sets, it is open, whence by the same reasoning as earlier $J = G$.
    Thus, since $G$ is compact, $\{W_k\}_{k=1}^{\infty}$ is an open cover of $G$, and as $W^k$ is increasing in $k$, it follows that $G = W^N$ for some $N$.
    Since $W \subseteq U^m$, $W^N \subseteq U^{mN}$, whence $U^{mN} = G$, as desired.
\end{proof}

\end{document}
